\chapter*{Resumen} 

Este trabajo consiste en la aplicación de metodologías ágiles a ALTEM, el cual es un proyecto de software de la Universidad Tecnológica de Bolívar propuesto por el departamento de Bienestar Universitario debido a la necesidad de llevar el control de la población estudiantil que cumple con ciertos factores de riesgo, este proyecto ayudaría al personal de Bienestar Universitario a prevenir la deserción estudiantil.

La función principal de este software es identificar a los estudiantes activos que presenten una condición de riesgo. Cuando un estudiante sea identificado, el software se encarga de disparar una alerta para que este estudiante sea intervenido por parte del equipo de consejería estudiantil y se le empiece un seguimiento a modo de bitácora. 
El software es capaz de generar un listado de estudiantes que cumplan con estas condiciones de riesgo, cuando se accede a uno de estos estudiantes, se crea una línea de tiempo en la cual se puede añadir información de las acciones y estrategias que se le han aplicado al estudiante desde el momento en el que fue disparada la alerta hasta que este sale de la situación de riesgo. 

Este software fue desarrollado inicialmente por Hernando Ariza durante el año 2016 y presentado como su trabajo de grado. 

Durante el año siguiente, el personal de Bienestar Universitario empezó a realizar pruebas en este software, con el fin de obtener retro-alimentación y decidir si el proyecto era sostenible como para continuar su financiación. 

A finales de ese año, se concluyó que el software podría cumplir con los objetivos por los cuales fue creado, pero presentaba ciertas falencias a nivel de funcionalidad, experiencia de usuario y requerimientos.

En consecuencia, a principios de 2018 Miguel Isaza decidió continuar con el mantenimiento y desarrollo de este proyecto, que ahora presentaba nuevos requerimientos funcionales y requería de reparaciones de bugs

Se diseñó un plan de trabajo el cual consistía en desarrollar las nuevas funcionalidades y mejorar los problemas que presentaba el software en ese momento utilizando las metodologías Agiles, realizar procesos de QA y que este entrase en periodo de prueba nuevamente por parte del personal de Bienestar Universitario para recibir nueva retro-alimentación. 
