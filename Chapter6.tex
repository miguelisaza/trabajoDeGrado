\chapter{Pruebas y resultados}
Luego de que se hiciera el despliegue, la aplicación empezó a ser usada de nuevo por el personal de consejería en el departamento de bienestar universitario de la Universidad Tecnológica de Bolívar. Los cuales ya venían realizando varios procesos de seguimiento estudiantil con ALTEM. 

\section{Retroalimentación de Consejería}
El 4 de Diciembre de 2019, hubo una reunión con el departamento de Observación Estudiantil de Bienestar Universitario, en donde se entrevistó a las consejeras acerca del estado actual de ALTEM y su satisfacción con las mejoras que se le hicieron al producto.

\subsection{Satisfacción General}
El personal de bienestar universitario se siente completamente satisfecho con el producto final ya que se cumplieron todos los requerimientos pactados y se solucionaron todos los problemas en la aplicación, además contemplan con agrado las mejoras en la interfaz y experiencia de usuario. 

\subsection{Confianza en ALTEM}
El personal de consejería de Bienestar Universitario mencionó que el sistema ahora es completamente funcional y que se encuentra mucho más estable ahora que en su versión anterior. Sin embargo, consideran que es un sistema no 100\% confiable debido a que desconocen el estado de robustez de la infraestructura que lo soporta. Consideran que sería bueno tener servidores de respaldo para asegurar la información de todos los casos activos que se están manejando a través de ALTEM. 

Por otra parte, también solicitaron la capacitación de una persona en el departamento de Tecnologías de Información para que conozca el sistema y sea capaz de darle soporte técnico en caso de que algún incidente ocurra.

\subsection{Consideraciones de Inversión}
El personal de consejería considera que ALTEM se encuentra en un punto de estabilidad bastante robusto, y que merece seguir siendo financiado para el complemento de todas sus funcionalidades. Definitivamente se considera que es mejor seguir invirtiendo en ALTEM que comprar otro producto que ofrezca las mismas soluciones.
También, se plantearon módulos que extenderían su funcionalidad hacia otros departamentos de la universidad.

\subsection{Plan de Trabajo a futuro}
El personal de consejería se encuentra motivado con ALTEM, por lo que planteó que se realizara un plan de trabajo a futuro para extender la funcionalidad de la aplicación a otros departamentos de la universidad, como por ejemplo a los profesionales de apoyo y a la dirección de programa, permitiendo que ellos tengan acceso a la plataforma para agregar anotaciones, observaciones y eventualidades en el archivo personal de los estudiantes. 

También se planteo la creación de reportes estadísticos con tópicos como:

\begin{itemize}
    \item Cual es el Programa con mas Atenciones
    \item Cuáles son los factores de riesgo más frecuentes
    \item Cuál es la estrategia más usadas
\end{itemize}

El personal de consejería tiene bastantes espectativas y esperan que el proyecto se siga desarrollando.
