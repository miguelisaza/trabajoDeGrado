\chapter{Abstract}

This work consists in the application of agile methodologies to ALTEM, which is a software project of the Universidad Tecnológica de Bolívar proposed by the Department of University Welfare due to the need to control the student population that meets certain risk factors, this project would help the psycologists of University Welfare to prevent student desertion.

The main function of this software is to identify active students who present a risk condition. When a student is identified, the software triggers an alert for this student in order to be intervened by the student counseling team and track its progress in a logbook. 

The software is capable of generating a list of students who meet these risk conditions, when you access one of these students in details, the software shows a timeline in which you can add information of actions and strategies that have been applied to the student since the alert was triggered until it leaves the risk situation. 

This software was initially developed by Hernando Ariza during 2016 and presented as his degree work. 

During the following year, University Welfare staff began testing this software, in order to get feedback and decide if the project was sustainable enough to continue its funding. 

At the end of that year, it was concluded that the software could meet the objectives for which it was created, but had certain shortcomings at the level of functionality, user experience and requirements.

Consequently, at the beginning of 2018 Miguel Isaza decided to continue with the maintenance and development of this project, which now presented new functional requirements and required bug fixes.

A work plan was designed which consisted of developing the new functionalities and improving the problems presented by the software at that time, performing QA processes so it can enter a in a trial period again by the University Welfare staff to receive new feedback. 