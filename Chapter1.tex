\chapter{Introducci\'on}
ALTEM es una aplicación de software cuya función principal es servir de apoyo a la gestión de Bienestar Estudiantil en lo referente al seguimiento de estudiantes en situaciones de interés, desde el momento de su inscripción, entrevista y posterior ingreso a clases. Las situaciones de interés mencionadas corresponden a eventos como la prueba académica, o condiciones de vulnerabilidad relacionadas al entorno socio- económico del estudiante, entre otras.

ALTEM se concibe entonces como una herramienta que le debe permitir a Bienestar Estudiantil identificar a los estudiantes que se encuentran en alguna de estas situaciones de interés, para iniciar los procedimientos del caso. Adicionalmente, el sistema debe permitir contar con un registro de las acciones e intervenciones que se han llevado a cabo sobre la situación de los estudiantes identificados. La meta final es evitar la posible deserción estudiantil.

\section{Justificación}
Toda institución académica llega a al punto de quiebre en el cual necesita un departamento de bienestar estudiantil, el cual se encarga de vigilar la calidad académica de los estudiantes. Cuando el volumen de estudiantes es muy grande, surge la necesidad de implementar sistemas de información que faciliten a los consejeros estudiantiles llevar un control de estos estudiantes. 

Al ser una necesidad inminente, se han venido implementando sistemas de información que nos brindan soluciones para esta problemática, como lo son XXXXX, YYYYY, ZZZZZ los cuales se venden y distribuyen como un producto comercial.

ALTEM surge como una alternativa a estos sistemas, cuyo valor agregado es su seguimiento en forma de bitácora, lo que hace que la experiencia de usuario sea considerablemente más agradable con respecto a sus iguales.

Durante el desarrollo de ALTEM, se introdujeron diversos patrones y diseños en la arquitectura del software que dificultaron la extensión de este al implementar nuevos requerimientos, además de la presencia de considerables fallas o bugs que complicaban la interacción con el usuario y en consecuencia afectaba negativamente la calidad general del software.

Este proyecto parte de la necesidad de aplicar las metodologías de mantenimiento ágil a la plataforma ALTEM, con el fin de mejorar falencias a nivel de funcionalidad y experiencia de usuario mencionadas anteriormente.

\section{Estado del arte}

(otros proyectos que se hayan hecho de lo mismo

\subsection{Investigaciones Internacionales}
\subsection{Investigaciones Nacionales}

\subsection{Antecedentes}
(antecedentes, otros proyectos y productos existentes que hagan lo mismo, de lo macro a lo micro)
\subsection{Marco Teórico}
(en qué teorías me baso para desarollar el proyecto)
\subsection{Marco Conceptual}
(conceptos y palabras claves que se deben aclarar para que el lector sepa, todo referenciable)
\subsection{Marco Legal}
(Leyes colombianas de Habeas Data y proteccion de datos sensibles)
\subsection{Marco Ético}
(Compromiso por parte de los desarolladores del compromiso con los datos sensibles manejados en las bases de datos)
\section{Objetivos}

\subsection{Objetivo General}
Desarrollar un plan de mejoramiento de la aplicación ALTEM para reactivar su uso por parte de los consejeros académicos del departamento de Bienestar Universitario en la Universidad Tecnológica de Bolívar.

\subsection{Objetivos Específicos}
\begin{itemize}
    \item Actualizar el algoritmo de autenticación de usuarios (de LDAP a SAVIO)
    \item Conectar, consumir y sincronizar los datos académicos y estudiantiles de ALTEM con el nuevo sistema BANNER de forma automatizada.
    \item Resolver las fallas (bugs) presentes en la aplicación.
    \item Hacer refactorización de los módulos que presentan problemas de calidad de código y arquitectura.
    \item Implementar los nuevos requerimientos solicitados el personal de consejería académica
\end{itemize}