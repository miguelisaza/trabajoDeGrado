\chapter{Conclusiones}
\section{Recepción del Proyecto}
ALTEM es un proyecto de la universidad Tecnológica de Bolívar que ha facilitado el proceso de atención a los estudiantes por parte de los consejeros del departamento de bienestar universitario, los cuales determinan que el proyecto ha sido satisfactorio y que el software se encuentra en un punto de usabilidad plena con buenas expectativas hacia el futuro. 

Se considera que ALTEM debe seguir siendo desarrollado ya que los costos de desarrollo de este proyecto son considerablemente inferiores al hecho de adquirir un nuevo producto que cumpla con los mismos requerimientos. 
Los consejeros se encuentran motivados y determinados a seguir utilizando la plataforma como herramienta principal para el seguimiento de sus estudiantes, con toda la disposición de realizar un nuevo plan de trabajo con requerimientos nuevos los cuales harían la aplicación aún más confiable y de usabilidad más integral, no solo para ellos sino para otros departamentos de la Universidad Tecnológica de Bolívar. 

En términos generales, el recibimiento por parte del departamento de Bienestar fue positivo y consideran totalmente viable la continuidad en la inversión del proyecto.

\section{Conclusiones del Autor}
El autor de este trabajo como desarrollador de software considera que la aplicación cuenta con una arquitectura fácil de mantener, pero existen componentes dentro de esta arquitectura que tienen dependencias las cuales se encuentran en estado de depreciación o están cerca de estarlo, por lo cual se recomienda que se haga una actualización del código fuente para que este pueda ser más fácil de mantener y tenga mayor apoyo por parte de la comunidad Open-Source en la que están basadas varias de sus dependencias.
Una de esas dependencias es el Framework sobre el cual se basa la API. El cual es una versión de Laravel muy cerca de quedar depreciada. Se considera que la mejor solución es migrar esa API a la nueva implementación ligera de Laravel conocida como Lumen, ya que tiene una orientación mas acertada al tipo de API que se desarrolló para este proyecto.

